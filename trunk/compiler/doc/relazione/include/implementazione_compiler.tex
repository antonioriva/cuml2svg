\section{Implementazione Compilatore}

Il compilatore è stato costruito con l'ausilio di javacc. Essendo il linguaggio
costituito da due tipi di file è stato necessario capire come poter effettuare
il parsing. La soluzione è stata quella di caricare (al momento della direttiva
di import) il file al volo e iniziare il parsing del file sospendendo
quello principale per poi riprenderlo successivamente; si è fatto in modo di
separare fortemente le due sezioni in modo da poter implementare una
funzionalità di solo checking (utile per il plugin eclipse).

Questa funzionalità è lampante se si analizza il listato del comando:

\begin{lstlisting}[ style={none}]
java -jar cUml2Svg.jar --help
\end{lstlisting}

oppure eseguendo la classe ``org.cuml2svg.compiler.Compiler''. Il risultato sarà:

\begin{lstlisting}[caption={Output dell'help da linea di comando}, style={none}]
cUml2Svg - generation of svg diagram
        from a coded rappresentation

  Usage: u2sc [option]
 
 --help         print this message
 
 --disable-warning, -dw disable warning messages
 --disable-error, -de   disable error messages
 --disable-notice, -dn  disable notice mesages
 
 --check, -c            only check syntax
 
 --input, -i            input layout file path
 --output, -o           output svg file path
 -t                     path path to the template folder
\end{lstlisting}

Analizzando le opzioni:

\begin{itemize}
  \item --help: utile per stampare a monitor il messaggio di aiuto;
  \item -dw,-de,-dn: per non stampare, nell'ordine, messaggi d'attenzione,
  d'errore, di notifica;
  \item -c: per effettuare solo i controllo del file e non generare l'output;
  \item -i: il file di input da processare (obbligatorio);
  \item -o: il file di output da processare (obbligatorio solo se non è presente
  -c);
  \item -t: è il percorso ai file di template per la generazione dell'svg, poter
  indicare dove sono questi file permette ad un utente avanzato di intervenire
  su questi file ed effettuare cambiamenti estetici alla visualizzazione del
  diagramma; se non indicato punta alla cartella ``./templates''.
\end{itemize} 


