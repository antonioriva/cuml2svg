\chapter{Considerazioni conclusive}

Questo progetto ci ha permesso di conoscere lo strumento JAVACC per la
creazione di nuovi linguaggi partendo da una grammatica. JAVACC è uno strumento
molto potente e permette di realizzare linguaggi di qualsiasi livello di
complessità senza troppi problemi.
Nell'ambito del progetto, lo sviluppo del plugin per l'ambiente Eclipse e la
generazione di grafica SVG attraverso l'uso del template engine Velocity sono
stati molto utili per apprendere le tecnologie attualmente sul mercato.
La scelta del formato SVG è stata guidata dal fatto che si tratta di un formato
aperto e che rappresenta uno standard del W3C ovvero di un'organizzazione che
definisce i formati che vengono utilizzati a livello mondiale. Ciò significa che
l'output del nostro generatore potrà essere facilmente importato da un qualsiasi
software per la grafica vettoriale ed eventualmente personalizzato secondo le
proprie esigenze.
Anche la scelta di realizzare un plugin per Eclipse è una scelta molto attuale,
utile a noi per imparare a conoscere questo ambiente che sta rapidamente
divenendo lo standard de facto fra gli ambienti di sviluppo grazie alla sua
potenza e flessibilità che gli permette di adattarsi ad una moltitudine di
esigenze anche completamente diverse dallo sviluppo software.


Al termine dello sviluppo dell'intero progetto sono stati analizzati i
tempi impiegati e confrontati con quelli previsti all'inizio durante la stesura
del diagramma Gantt. Possiamo dire che per le attività principali di sviluppo
del compilatore i tempi sono stati rispettati quasi completamente, mentre per la
creazione del plugin e per l'esportazione in formato SVG i tempi sono stati
leggermente superiori a causa di alcuni piccoli inconvenienti tecnici causati
dall'approccio con nuovi strumenti non sempre perfettamente documentati.
Possiamo comunque dire di essere stati nei tempi previsti per quanto riguarda il
termine delle attività complessive.


Quanto realizzato nell'ambito di questo progetto è un semplice
strumento
che per noi può essere utile ad altri, per questo si è scelto di pubblicare
tutto con licenza Open Source su un ambiente condiviso quale Google Code
\cite{google_code_website:10}, nella speranza che qualcuno lo noti e lo estenda
rendendolo ancora più utile per lui e per l'intera comunità degli sviluppatori.
