\chapter{Il linguaggio}

Il linguaggio che stiamo proponendo è strutturato in 2 parti, e quindi due tipi di
file, che corrispondono al concetto di modello e layout; non a caso l'estensione
del primo è u2sm e il secondo è u2sl.

Il file modello sarà il contenitore delle definizione delle classi mentre nel
file layout il programmatore indicherà l'aspetto grafico del diagramma che si va
a generare.

\section{Model}

Il modello conterrà quindi tutte le informazioni riguardo il contenuto/i dati
degli elementi che sarà possibile inserire in un diagramma in un secondo
momento. 

In un linguaggio ad oggetti uno degli strumenti fondamentali è la suddivisione 
delle classi sotto forma di package. Nella nostra proposta un file di modello
può contenere o una lista di classi (non associate ad un package) oppure una
lista di package.

La definizione del package (come un po' tutto il nostro linguaggio) è 
ispirata alla sintassi dei linguaggi java like.

\begin{lstlisting}[caption={Dichiarazione di package}, style={model}]
package nome.package{
	...  	
}
\end{lstlisting}

Nella definizione della classe è invece dichiarato in modo esplicito la
suddivisione tra i componenti base di una classe; sono quindi esplicitati quali
sono gli attributi, i metodi e le relazioni. L'aggiunta delle relazioni rispetto
ad una classica definizione di classe è necessaria per poter rappresentare nel
diagramma le frecce di collegamento.


\begin{lstlisting}[caption={Dichiarazione di classe}, style={model}]
class NomeClasse{
	relations{
		...
	}
	attributes{
		...
	}
	methods{
		...
	}
}
\end{lstlisting}

Analizziamo ora il contenuto delle varie sezioni.

\subsection{Relazioni}

Nella sezione delle relazioni
viene appunto dichiarata una lista di definizione che contiene il tipo della 
ralazione, una lista di classi (complete di package) 
con cui creare una relazione del tipo scelto
per ogni classe 3 parametri opzionali (che possono anche essere vuoti) in cui
il primo parametro e l'ultimo possono essere utilizzati per la definizione delle
cardinalità, mentre il parametro opzionale puù essere utilizzato per definire
una descrizione della relazione; queste stringhe andranno a posizionarsi sul 
disegno delle frecce.

\begin{lstlisting}[caption={Dichiarazione di relazione}, style={model}]
relations{
	extend nome.package.NomeClasse ("(1,*)" , 
				"descrizione relazione", "(1,1)" );
	use nome.package.NomeClasse1, nome.package.NomeClasse2;
}
\end{lstlisting}

La lista dei possibili tipi di relzione sono:
\begin{itemize}
  \item{use (uso);}
  \item{extends (ereditarietà);}
  \item{associate (associazione);}
  \item{include (inclusione);}
  \item{realize (realizzazione);}
  \item{depend (dipendenza);}
  \item{composed (composizione);}
\end{itemize}

\subsection{Attributi}

Gli attributi vengono dichiarati come visibilità:
\begin{itemize}
  \item public
  \item privare
  \item protected
\end{itemize}

tipo (opzionale), nome attributo e valore di default (opzionale); la sintassi è
quella inidicata nell'esempio sottostante.

\begin{lstlisting}[caption={Dichiarazione di attributi}, style={model}]
attributes{
	public int attr1=5;
	public attr2;
	private attr3;
}
\end{lstlisting}

\subsection{Metodi}

Anche in questo caso la definizione dei metodi è del tutto simile alla
difinizione della segnatura di un metodo in java.

\begin{lstlisting}[caption={Dichiarazione di metodi}, style={model}]
methods{
	public nomeMetodo1(int arg1=10, String arg2,arg3);
	private nomeMetodo2(arg1);
	nomeMetodo3();
}
\end{lstlisting}

Nell'esempio di definizione sopra riportato si possono contraddistinguere varie 
caratteristiche aggiuntive del nostro linguaggio; un metodo è caratterizzato
oltre che dal suo nome anche dal tipo di visibilità.

Gli attributi possono essere definiti sia con l'assegnazione di un tipo che con
la definizione del solo nome, c'è anche la possibilità di assegnare un valore di
default al parametro del metodo (caratteristica disponibile in alcuni linguaggi
tipo il PHP).
Possiamo notare il fatto che è possibile omettere gran parte degli attributi
perchè l'idea è che deve essere obbligatorio definire solo gli elementi minimi
per poi poter indicare, opzionalmente, solo ciò che si vuole rappresentare nel diagramma.