\chapter{Introduzione}

Per l'esame di \emph{progetto di linguaggi e compilatori} abbiamo deciso di
basarci sul progettino consegnato per l'esame \emph{linguaggi e compilatori}
tenuto lo scorso semestre. \\*
L'obiettivo del progettino è stata la definizione del linguaggio per la 
descrizione di diagrammi UML. In particolar modo si era prestata attenzione al
controllo di dichiarazione/uso/non redichiarazione
delle classi e a controlli di non re-dichiarazione di package.




\section{cUml2Svg - Coded UML to SVG} 
Il progetto da noi sviluppato si chiama \emph{cUml2Svg}. \\*
L'acronimo cUml2Svg sta per ``\emph{Coded UML to SVG}``; 
il progetto consiste nello sviluppare un traduttore che traduca da una
rappresentazione in codice dei digrammi UML delle classi alla rappresentazione
grafica tipica dell'UML.\\*
La rappresentazione grafica da noi scelta e sulla quale abbiamo basato il nostro
progetto è l'SVG.\\*
L'SVG (Scalable Vector Graphics), indica una tecnologia in grado di visualizzare oggetti 
di grafica vettoriale e, pertanto, di gestire immagini scalabili dimensionalmente.
Più specificamente si tratta di un linguaggio derivato dall'XML, che si pone l'obiettivo 
di descrivere figure bidimensionali statiche o animate. \\*
In Particolare SVG permette di trattare tre tipi di oggetti grafici:

\begin{itemize}
  \item forme geometriche, cioè linee costituite da segmenti di retta, curve e
  aree delimitate da linee chiuse;
  \item immagini della grafica raster e immagini digitali;
  \item testi esplicativi, eventualmente cliccabili.
\end{itemize} 


Il progetto cUml2Svg è completo anche di interfaccia grafica; è stata costruita
come estensione del popolare ambiente di sviluppo Eclipse~\cite{eclipse_website:1}

